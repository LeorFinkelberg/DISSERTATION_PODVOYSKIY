\chapter*{Словарь терминов}             % Заголовок
\addcontentsline{toc}{chapter}{Словарь терминов}  % Добавляем его в оглавление

\textbf{Простой граф\footnote{Вместо термина <<простой граф>> часто употребляется термин <<граф>>}} : Совокупность двух множеств -- непустого множества вершин $ V $ и множества ребер $ E $ неупорядоченных пар различных элементов множества $ V $ называют простым графом
\begin{align*}
	G(V, E) = <V, E>, \quad V \neq \emptyset,\ E \in V \times V.
\end{align*}

\textbf{Полный граф} : Граф $ G(V, E) $ называют полным, если в нем любые две вершины смежны, то есть $ \forall u, v \in V, \ (u, v) \in E $. Полный граф определяется числом своих вершин и обозначается $ K_n $, где $ n $ -- мощность множества вершин $ V $

\textbf{Смежные вершины графа} : Вершины называют смежными, если они инцидентны соединяющему их ребру

\textbf{Смежные ребра графа} : Два ребра, инцидентные одной вершине, называют смежными

\textbf{Степень (валентность) вершины} : Количество ребер, инцидентных вершине $ v $, называют \emph{степенью} или \emph{валентностью} вершины $ v $ и обозначают $ d(v) $

\textbf{Изоморфизм графов} : Если графы являются изоморфными, то они имеют одинаковое количество вершин и ребер, а также одинаковое число вершин одной степени. Обратное неверно
