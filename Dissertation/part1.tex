\chapter{Ключевые положения эврестического подхода к описанию задачи механики деформируемого твердого тела в терминах графов}\label{ch:ch1}

\emph{Инвариант графа}, как известно, это функция на графе, которая сопоставляет каждому графу некоторое значение, служащее мерой подобия структуры любых двух графов, обладающих свойством изоморфности.

Инварианты не всегда подходят для установления изоморфизма графов, потому как графы с одинаковыми значениями инвариантов могут оказаться неизоморфными. Однако в тех сценариях, когда большее значение имеет не свойство инварианта устанавливать изоморфизм графов, а его способноть различать \emph{неизоморфные} графы, инварианты могут быть очень полезны.

Инварианты графов, определяемые как функции от расстояний между вершинами графа, находят многочисленные приложения в областях, для которых отношения между объектами могут быть представленны по схеме <<сущность - связь>>. Например,  в качестве моделей структурных формул химических соединений традиционно используются молекулярные графы \cite{Stankevich-graph-struc-chem:1988, Gutman:1986}.

{\color{red}Численные инварианты, которые не учитывают <<метрические>> особенности задачи, но учитывают структурные, называют \emph{топологическими индексами}. Как отмечается в работе \cite{Dobrinin-viner:1998}, успех применения индексов для прогнозирования свойств соединений основан на том, что молекулярные графы с <<похожей>> структурой (и с близкими значениями индекса как мерой подобия структур) соответствуют соединениям со сходными свойствами.}

Введем в рассмотрение понятие \emph{графа} $ G(V, E) $ как совокупности непустого множества \emph{вершин} $ V $ и множества \emph{ребер} $ E $ (множества \emph{неупорядоченных} пар различных элементов множества $ V $)
\begin{align*}
	G(V, E) = <V, E>, \qaud V \neq \emptyset, \ E \in V \times V.
\end{align*}

\underline{Замечание}: В настоящем разделе и далее речь идет о \emph{неориентированных} графах, то есть о таких графах, в которых порядок обхода вершин не имеет значения.

\underline{Замечание}: Вершины \emph{ориентированного} графа -- графа, элементы множества ребер $ E $ которого являются упорядоченными парами (парами, в которых фиксирован порядок элементов) -- принято называть \emph{узлами}, а элементы множества $ E $ -- \emph{дугами}. Во избежании терминологической путаницы следует подчеркнуть, что в настоящем разделе и далее термин <<узел>> резервируется для описания узлов конечно-элементой модели и не имеет никакого отношения к ориентированным графам.

\section{Описание геометрии деформируемого тела в терминах метрических и топологических индексов на неориентированных графах}\label{sec:ch1/sec1}




\section{Инвариант на взвешенном графе геометрически-силового представления}\label{sec:ch1/sec2}



\FloatBarrier
