\chapter{Основные положения концепции исчерпания несущей способности нагруженного объекта}\label{ch:ch1}

Базовые расчетные положения \emph{концепции исчерпания прочностных характеристик нагруженных силовых элементов конструкций} (КИПХ), связанные с вопросами прогнозирования усталостной долговечности объектов, работающих в условиях \emph{нестационарного} стохастического нагружения, были рассмотренны в общей постановке в работах автора \cite{podvoyskiy-ipmash:2009}, \cite{podvoyskiy-fund-prob:2010}, \cite{podvoyskiy-mami:2010}, \cite{podvoyskiy-sstu:2010} и \cite{podvoyskiy-sstu:2011}.

Модификации КИПХ применительно к случаям нагружения силовых элементов конструкции \emph{стационарными гауссовскими} и \emph{негауссовскими} стохастическими процессами автор развивал в работах \cite{podvoyskiy-vestnik:2013} и \cite{podvoyskiy-mami:2012}.

Введем в рассмотрение величину, которую условимся обозначать через $ \sigma_0 $, описывающую априорную информацию об объекте, а именно информацию о текущем техническом состоянии нагруженного участка магистрального трубопровода без дефектов.

По мере приложения нагрузки контур дефекта развивается. По мере развития контура дефекта уменьшается значение величины $ \sigma_0 $. Таким образом, исчерпание несущей способности\footnote{Под несущей способностью условимся понимать предельную нагрузку, которую способен выдерживать объект без перехода в предельное состояние} нагруженного участка магистрального трубопровода, ослабленного дефектами геометрии с потерей металла, можно описать в терминах деградационных процессов.
